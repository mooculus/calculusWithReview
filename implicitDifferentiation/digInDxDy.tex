\documentclass{ximera}

\input{../preamble.tex}

\outcome{Implicitly differentiate expressions.}
\outcome{Find the equation of the tangent line for curves that
  are not plots of functions.}
\outcome{Understand how changing the variable changes how we take
  the derivative.}
\outcome{Understand the derivatives of expressions that are not
  functions or not ``solved for $y$''.}

\title[Dig-In:]{Finding $\frac{dx}{dy}$}

\begin{document}
\begin{abstract}
In this section we differentiate equations without expressing them in
terms of a single variable.
\end{abstract}
\maketitle

\section{Turn it around}
	When we see an equation, like $x + y^3 - y = 1$, $x$ and $y$ are both quantities that change according to that formula.  We typically consider $y$ as a function of $x$, but we 
	can also consider $x$ as a function of $y$.

	\begin{example}
		The equation $x + y^3 - y = 1$ defines $x$ implicitly as a function of $y$.  Find $\dd[x]{y}$.
		
		\begin{explanation}
			\begin{align*}
				\dd{y} (x + y^3- y ) &= \dd{y} (1)\\
				\dd[x]{y} + 3y^2 - 1 &= 0\\
				\dd[x]{y} = 1 - 3y^2
			\end{align*}
		\end{explanation}
	\end{example}

	We know that at points with $\dd[x]{y} = 0$, our graph has horizontal tangent lines. Something similar can be said for vertical tangent lines.
	\begin{theorem}[Finding Vertical Tangent Lines]\index{vertical tangent lines}
		If a point on the graph of an equation has $\dd[y]{x}= 0$, then the tangent line at that point is vertical.
	\end{theorem}

	\begin{example}
		Find the points on the graph of $x + y^3 - y = 1$ where the tangent line is vertical.
		
		\begin{explanation}
			We have already seen that $\dd[x]{y}= 1-3y^2$.
			\begin{align*}
				\dd[x]{y} &= 0\\
				1 -3y^2 &= 0\\
				3y^2 &= 1\\
				y^2 &= \answer{\frac{1}{3}}\\
				y &= \pm \answer{\frac{1}{\sqrt{3}}}
			\end{align*}
			Plugging these back into the equation:
			\begin{align*}
				y &= \frac{1}{\sqrt{3}}\\
				x &= 1 + y - y^3\\
					&= 1 + \frac{2}{3\sqrt{3}}
			\end{align*}
				One point is $\left( 1 + \frac{2}{3\sqrt{3}} , \frac{1}{\sqrt{3}} \right)$.
				Plugging in the other value of $y$ gives the other point: $\left(  1 - \frac{2}{3\sqrt{3}} , -\frac{1}{\sqrt{3}}\right)$.
		\end{explanation}
	\end{example}


\end{document}
