\documentclass{ximera}

\input{../../preamble.tex}

\outcome{Define linear approximation as an application of the tangent to a curve.}
\outcome{Find the linear approximation to a function at a point and use it to approximate the function value.}
\outcome{Compute differentials.}
%\outcome{Contrast the notation and meaning of \d{y} versus \Delta y.}

\author{Bobby Ramsey}

\begin{document}
\begin{exercise}

	Consider the function $f(x) = \tan(x)$.

	The linear approximation of $f$ at $a=\frac{\pi}{4}$ is 
	\[ L(x) = \answer{2(x-\frac{\pi}{4})+1}. \]

	The graph of the linear approximation function $L(x)$ is a \wordchoice{\choice[correct]{line}\choice{parabola} \choice{circle}} which is
		\wordchoice{\choice{secant}\choice[correct]{tangent} \choice{unrelated}} to the graph of $f$.
	\begin{exercise}
		The formula for the differential $df$ is given by:
		\[ df = \answer{\sec^2(x)} dx \]					
		
		\begin{exercise}
			If $x$ changes to $44^\circ$, (from the value of $a$ specified above) use differentials to estimate the change in $f$.
			\[ df = \answer{\frac{-\pi}{90}} \]
			
			\begin{exercise}
				Use this to estimate \[\tan\left( 44^\circ \right) \approx \answer{1-\frac{\pi}{90}} \]
				
				\begin{exercise}
					This estimate is:
					\begin{multipleChoice}
						\choice{an overestimate.}
						\choice[correct]{an underestimate.}
						\choice{the exact value of $\tan\left( 44^\circ \right)$.}
						\choice{impossible to determine.}
					\end{multipleChoice}
				\end{exercise}
			\end{exercise}
		\end{exercise}
	\end{exercise}
\end{exercise}
\end{document}