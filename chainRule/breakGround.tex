\documentclass{ximera}

\input{../preamble.tex}

\outcome{Recognize a composition of functions.}
\outcome{Take derivatives of compositions of functions using the chain rule.}
\outcome{Take derivatives that require the use of multiple derivative rules.}
\outcome{Use the chain rule to calculate derivatives from a table of values.}
\outcome{Understand rate of change when quantities are dependent upon each other.}
\outcome{Use order of operations in situations requiring multiple derivative rules.}
\outcome{Apply chain rule to relate quantities expressed with different units.}

\title[Break-Ground:]{An unnoticed composition}

\begin{document}
\begin{abstract}
Two young mathematicians discuss the chain rule.
\end{abstract}
\maketitle

Check out this dialogue between two calculus students (based on a true
story):

\begin{dialogue}
\item[Devyn] Riley! Something is bothering me. 
\item[Riley] What is it?
\item[Devyn] I have broken calculus. 
\item[Riley] ..
\item[Devyn] ..
\item[Devyn]..
\end{dialogue}

This problem that Riley and Devyn are having is somewhat subtle. 

\begin{question}
  What mistake is being made?

\begin{multipleChoice}
\choice{Riley did not plot $\sin(\theta)$.}
\choice{Riley did not take the derivative correctly.}
\choice[correct]{Riley was working in degrees, not radians.}
\choice{$\cos(0)\ne 1$}
\choice{Riley computed the slope incorrectly.}
\end{multipleChoice}
\begin{feedback}
  In calculus, we work with radians. Working in degrees will produce
  erroneous answers.
\end{feedback}
\end{question}


%\input{../leveledQuestions.tex}


\end{document}
