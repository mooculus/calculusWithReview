\documentclass{ximera}

\input{../preamble.tex}

\outcome{Understand the Squeeze Theorem and how it can be used to find limit values.}
\outcome{Calculate limits using the Squeeze Theorem.}

\title[Dig-In:]{The Squeeze Theorem}

\begin{document}
\begin{abstract}
The Squeeze theorem allows us to exchange difficult functions for easy functions.
\end{abstract}
\maketitle

In mathematics, sometimes we can study complex functions by exchanging
them for simplier functions. The \textit{Squeeze Theorem} tells us one
situation where this is possible.

\begin{theorem}[Squeeze Theorem]\index{Squeeze Theorem}
  Suppose that
  \[
  g(x) \le f(x) \le h(x)
  \]
  for all $x$ close to $a$ but not necessarily equal to $a$. If
  \[
  \lim_{x\to a} g(x) = L = \lim_{x\to a} h(x),
  \] 
  then $\lim_{x\to a} f(x) = L$.
\end{theorem}

\begin{question}
  I'm thinking of a function $f$. I know that for all $x$
  \[
  0 \le f(x) \le x^2.
  \]
  What is $\lim_{x\to 0} f(x)$?
  \begin{multipleChoice}
    \choice{$f(x)$}
    \choice{$f(0)$}
    \choice[correct]{$0$}
    \choice{impossible to say}
  \end{multipleChoice}
\end{question}


\begin{example}
An continuous function $f$ satisfies the property that $8x-13 \leq f(x) \leq x^2+2x-4$.
What is $\lim_{x\to 3} f(x)$?
	\begin{explanation}
		\[ \lim_{x\to 3} \left(8x-13\right) = \answer[given]{11}.\]		
		\[ \lim_{x\to 3} \left(x^2+2x-4\right) = \answer[given]{11}.\]
		Then \[\lim_{x\to 3} f(x) = \answer[given]{11}.\]
	\end{explanation}
\end{example}




\begin{example}
Consider the function
\[
f(x) = 
\begin{cases}
\sqrt[5]{x}\sin\left(\frac{1}{x}\right) & \text{if $x \ne 0$,}\\
0 & \text{if $x = 0$,}
\end{cases}
\]
\begin{image}
\begin{tikzpicture}
	\begin{axis}[
            domain=-.2:.2,    
            samples=500,
            width=6in,
            height=3in,
            axis lines =middle, xlabel=$x$, ylabel=$y$,
            yticklabels = {}, 
            every axis y label/.style={at=(current axis.above origin),anchor=south},
            every axis x label/.style={at=(current axis.right of origin),anchor=west},
            clip=false,
          ]
	  \addplot [very thick, penColor, smooth, domain=(-.2:-.02)] {abs(x)^(1/5)*sin(deg(1/x))};
          \addplot [very thick, penColor, smooth, domain=(.02:.2)] {x^(1/5)*sin(deg(1/x))};
	  \addplot [color=penColor, fill=penColor, very thick, smooth,domain=(-.02:.02)] {abs(x)^(1/5)} \closedcycle;
          \addplot [color=penColor, fill=penColor, very thick, smooth,domain=(-.02:.02)] {-abs(x)^(1/5)} \closedcycle;
        \end{axis}
\end{tikzpicture}
%% \caption[A continuous function.]{A plot of
%% \[
%% f(x)=
%% \begin{cases}
%% \sqrt[5]{x}\sin\left(\frac{1}{x}\right) & \text{if $x \ne 0$,}\\
%%  0 & \text{if $x = 0$.}
%% \end{cases}
%% \]
%% }
\end{image}
Is this function continuous at $x=0$?
\begin{explanation}
We must show that $\lim_{x\to 0} f(x) = \answer[given]{0}$. Note
\[
-|\sqrt[5]{x}|\le f(x) \le |\sqrt[5]{x}|.
\]
Since
\[
\lim_{x\to 0} -|\sqrt[5]{x}| = \answer[given]{0} = \lim_{x\to 0}|\sqrt[5]{x}|,
\]
we see by the Squeeze Theorem, Theorem, that
$\lim_{x\to 0} f(x) = \answer[given]{0}$. Hence $f(x)$ is continuous.

Here we see how the informal definition of continuity being that you
can ``draw it'' without ``lifting your pencil'' differs from the
formal definition.
\end{explanation}
\end{example}



\end{document}
