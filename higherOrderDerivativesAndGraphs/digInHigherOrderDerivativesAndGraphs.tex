\documentclass{ximera}

\input{../preamble.tex}

\outcome{Use the first derivative to determine whether a function is increasing or decreasing.}
\outcome{Define higher order derivatives.}
\outcome{Compare differing notations for higher order derivatives.}
\outcome{Identify the relationships between the function and its first and second derivatives.}


\title[Dig-In:]{Higher order derivatives and graphs}

\begin{document}
\begin{abstract}
 Here we look at graphs of higher order derivatives.   
\end{abstract}
\maketitle

An important application of derivatives is in determining when a function is increasing or decreasing.
\begin{definition}\index{increasing}\index{decreasing}
	A function is \emph{increasing} on an interval $I$ if, for all $a$ and $b$ in the interval with $a<b$, then $f(a) < f(b)$.
	A function is \emph{decreasing} on an interval $I$ if, for all $a$ and $b$ in the interval with $a<b$, then $f(a) > f(b)$.
\end{definition}
\begin{example}
The following is the graph of $y=f(x)$.  On what intervals is $f$ increasing?  On what intervals is $f$ decreasing?
  \begin{image}
  \begin{tikzpicture}
	\begin{axis}[
            xmin=-2.1,xmax=3.1,ymin=-3,ymax=4,
            axis lines=center,
            width=6in,
            height=3in,
            every axis y label/.style={at=(current axis.above origin),anchor=south},
            every axis x label/.style={at=(current axis.right of origin),anchor=west},
          ]        
          \addplot [very thick,penColor,smooth, domain=(-2:3)] {(x^4)/4-(x^3)/3-x^2)};
        \end{axis}
  \end{tikzpicture}
  \end{image}
  \begin{explanation}
  	As we move along the graph from left to right, we see that the $y$-values are getting smaller until we get to $x=-1$.  Between $x=-1$ and $x=0$ the
	$y$-values are getting larger, then between $x=0$ and $x=2$ they get smaller again.  Finally, after $x=2$ they start growing.
	
	$f$ is increasing on: $(-1,\answer{0})$ and $(\answer{2},\infty)$.  $f$ is decreasing on $(-\infty, \answer{-1})$ and $(\answer{0}, \answer{2})$.
  \end{explanation}	
\end{example}

Let's look at each of the intervals from the last example separately.
On $(-\infty,-1)$, look at the slopes of the tangent lines.  Is there anything that all of those slopes have in common?
They are all NEGATIVE!  Look in the interval $(0,2)$.  The slopes are all negative in that interval as well.
What about in the interval $(1,0)$?  Those slopes are all POSITIVE.  Same for the interval $(2,\infty)$.
Notice that the positive slopes occurred when the function was increasing and the negative slopes occurred
when the function was decreasing?  That was no accident.

Since the derivative gives us a formula for the slope of a tangent
line to a curve, we can gain information about a function purely from
the sign of the derivative.  In particular, we have the following theorem
\begin{theorem}
  If $f$ is differentiable on an interval, then
\begin{itemize}
\item $f'(x)>0$ on that interval whenever $f$ is increasing as $x$
  increases on that interval.
\item $f'(x)<0$ on that interval whenever $f$ is decreasing as $x$
  increases on that interval.
\end{itemize}
\end{theorem}
\begin{question}
  Below we have graphed $y=f(x)$:
  \begin{image}
  \begin{tikzpicture}
	\begin{axis}[
            xmin=-2,xmax=2,ymin=-8,ymax=8,
            axis lines=center,
            width=6in,
            height=3in,
            every axis y label/.style={at=(current axis.above origin),anchor=south},
            every axis x label/.style={at=(current axis.right of origin),anchor=west},
          ]        
          \addplot [very thick,penColor,smooth, domain=(-2:2)] {x^3+x^2-2*x)};
        \end{axis}
  \end{tikzpicture}
  \end{image}
  Is the first derivative positive or negative on the interval $-1<x<1/2$?
  \begin{prompt}
    \begin{multipleChoice}
      \choice{Positive}
      \choice[correct]{Negative}
    \end{multipleChoice}
  \end{prompt}
\end{question}

\begin{question}
  Below we have graphed $y=f'(x)$:
  \begin{image}
  \begin{tikzpicture}
	\begin{axis}[
            xmin=-2,xmax=2,ymin=-8,ymax=8,
            axis lines=center,
            width=6in,
            height=3in,
            every axis y label/.style={at=(current axis.above origin),anchor=south},
            every axis x label/.style={at=(current axis.right of origin),anchor=west},
          ]        
          \addplot [very thick,penColor,smooth, domain=(-2:2)] {x^3+x^2-2*x)};
        \end{axis}
  \end{tikzpicture}
  \end{image}
  Is the graph of $f(x)$ increasing or decreasing as $x$ increases on
  the interval $-1<x<0$?
  \begin{prompt}
    \begin{multipleChoice}
      \choice[correct]{Increasing}
      \choice{Decreasing}
    \end{multipleChoice}
  \end{prompt}
\end{question}

We call the derivative of the derivative the \dfn{second
  derivative}, the derivative of the derivative of the derivative the
\dfn{third derivative}, and so on. We have special notation for
higher derivatives, check it out:
\begin{description}
\item[First derivative:] $\ddx f(x) = f'(x) = f^{(1)}(x)$.
\item[Second derivative:] $\dd[~^2]{x^2} f(x) = f''(x) = f^{(2)}(x)$.
\item[Third derivative:] $\dd[~^3]{x^3} f(x) = f'''(x) = f^{(3)}(x)$.
\end{description}

We use the facts above in our next example.

\begin{example}
  Here we have unlabeled graphs of $f$, $f'$, and $f''$:
  \begin{image}
  \begin{tikzpicture}
	\begin{axis}[
            xmin=-2,xmax=2,ymin=-8,ymax=8,
            axis lines=center,
            ticks=none,
            width=6in,
            height=3in,
            every axis y label/.style={at=(current axis.above origin),anchor=south},
            every axis x label/.style={at=(current axis.right of origin),anchor=west},
          ]        
          \addplot [very thick,penColor,smooth, domain=(-2:2)] {x^3+.3*x^2-2*x)};
          \addlegendentry{$A$};
          \addplot [very thick, dashed,penColor,smooth, domain=(-2:2)] {3*x^2+2*.3*x-2)};
          \addlegendentry{$B$};
          \addplot [very thick, dotted,penColor,smooth, domain=(-2:2)] {6*x+2*.3)};
          \addlegendentry{$C$};
        \end{axis}
  \end{tikzpicture}
  \end{image}
  Identify each curve above as a graph of $f$, $f'$, or $f''$.
  \begin{explanation} 
    Here we see three curves, $A$, $B$, and $C$. Since $A$ is
    \wordChoice{\choice{positive} \choice{negative}
      \choice[correct]{increasing} \choice{decreasing}} when $B$ is
    positive and
    \wordChoice{\choice{positive}\choice{negative}\choice{increasing}\choice[correct]{decreasing}}
    when $B$ is negative, we see
    \[
    A'=B.
    \]
    Since $B$ is increasing when $C$ is
    \wordChoice{\choice[correct]{positive}\choice{negative}\choice{increasing}
      \choice{decreasing}} and decreasing when $C$ is
    \wordChoice{\choice{positive}\choice[correct]{negative}\choice{increasing}\choice{decreasing}}, we see
    \[
    B'=C.
    \]
    Hence $f=A$, $f'=B$, and $f''=C$.
  \end{explanation}
\end{example}




\begin{example}
    Here we have unlabeled graphs of $f$, $f'$, and $f''$:
    \begin{image}
      \begin{tikzpicture}
	\begin{axis}[
            domain=-4:4,
            ticks=none,
            ymax=2, ymin=-2,
            xmax=4, xmin=-4,
            axis lines =middle,
            every axis y label/.style={at=(current axis.above origin),anchor=south},
            every axis x label/.style={at=(current axis.right of origin),anchor=west},
            width=6in,
            height=3in,
          ]
          \addplot [very thick, penColor,smooth,samples=100] {2/(.75*sqrt(2*pi))*exp(-((x)^2)/(2*.75^2)) *(-x)/(.75^2)};
          \addlegendentry{$A$};
          \addplot [very thick, dashed,penColor,smooth,samples=100] {2*gauss(0,.75)};
          \addlegendentry{$B$};
          \addplot [very thick, dotted,penColor,smooth,samples=100] {2/(.75*sqrt(2*pi))*exp(-((x)^2)/(2*.75^2)) *(-1)/(.75^2) +
            2/(.75*sqrt(2*pi))*exp(-((x)^2)/(2*.75^2)) *(x^2)/(.75^4)};
          \addlegendentry{$C$};
        \end{axis}
          \end{tikzpicture}
\end{image}
    Identify each curve above as a graph of $f$, $f'$, or $f''$.
      \begin{explanation} 
        Here we see three curves, $A$, $B$, and $C$. Since $B$ is
        \wordChoice{\choice{positive}\choice{negative}\choice[correct]{increasing}\choice{decreasing}} when $A$
        is positive and
        \wordChoice{\choice{positive}\choice{negative}\choice{increasing}\choice[correct]{decreasing}} when $A$
        is negative, we see
        \[
        B'=A.
        \]
        Since $A$ is increasing when $C$ is
        \wordChoice{\choice[correct]{positive}\choice{negative}\choice{increasing}\choice{decreasing}}
          and decreasing when $C$ is
          \wordChoice{\choice{positive}\choice[correct]{negative}\choice{increasing}\choice{decreasing}}, we
          see
        \[
        A'=C.
        \]
        Hence $f=\answer[given]{B}$, $f'=\answer[given]{A}$, and
        $f''=\answer[given]{C}$.
      \end{explanation}
\end{example}

\begin{example}
  Here we have unlabeled graphs of $f$, $f'$, and $f''$:
  \begin{image}
  \begin{tikzpicture}
	\begin{axis}[
            xmin=-6.75,xmax=6.75,ymin=-1.5,ymax=1.5,
            axis lines=center,
            ticks=none,
            width=6in,
            height=3in,
            every axis y label/.style={at=(current axis.above origin),anchor=south},
            every axis x label/.style={at=(current axis.right of origin),anchor=west},
          ]        
          \addplot [very thick, penColor, samples=100,smooth, domain=(-6.75:6.75)] {-sin(deg(x))};
          \addlegendentry{$A$};
          \addplot [very thick, dashed,penColor, samples=100,smooth, domain=(-6.75:6.75)] {cos(deg(x))};
          \addlegendentry{$B$};
          \addplot [very thick, dotted,penColor, samples=100,smooth, domain=(-6.75:6.75)] {sin(deg(x))};
          \addlegendentry{$C$};
        \end{axis}
  \end{tikzpicture}
  \end{image}
  Identify each curve above as a graph of $f$, $f'$, or $f''$.
  %One is of $f$, another is of $f'$ and a third is of $f''$.  Explain
  %what strategies you could use to identify which graph corresponds
    \begin{explanation} %%BADBAD Need Dropdown
    Here we see three curves, $A$, $B$, and $C$. Since $C$ is
    \wordChoice{\choice{positive}\choice{negative}\choice[correct]{increasing}\choice{decreasing}} when $B$ is
    positive and \wordChoice{\choice{positive}\choice{negative}\choice{increasing}\choice[correct]{decreasing}}
    when $B$ is negative, we see
    \[
    C'=B.
    \]
    Since $B$ is increasing when $A$ is
    \wordChoice{\choice[correct]{positive}\choice{negative}\choice{increasing}\choice{decreasing}} and
    decreasing when $A$ is
    \wordChoice{\choice{positive}\choice[correct]{negative}\choice{increasing}\choice{decreasing}}, we see
    \[
    B'=A.
    \]
    Hence $f=\answer[given]{C}$, $f'=\answer[given]{B}$, and
    $f''=\answer[given]{A}$.
  \end{explanation}
\end{example}


\end{document}
