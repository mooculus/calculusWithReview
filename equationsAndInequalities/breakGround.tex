\documentclass{ximera}

\input{../preamble.tex}

\outcome{Solve equations.}
\outcome{Solve inequalities.}

\title[Break-Ground:]{  }

\begin{document}
\begin{abstract}
  Two young mathematicians examine an equation.
\end{abstract}
\maketitle

Check out this dialogue between two calculus students (based on a true
story):

\begin{dialogue}
	\item[Devyn] Riley, I have a question.
	\item[Riley] What's on your mind?
	\item[Devyn] Suppose we have three exams.  
	\item[Riley]  Ok. 
	\item[Devyn] On the first, I scored 74/100, and on the second I scored 84/100.
	\item[Riley] Not bad!  What's the question, though?
	\item[Devyn] What do I need to get on the third exam to average an 82\% overall?
	\item[Riley] To find the average of three numbers, you add them and divide the sum by three.
	\item[Devyn] Yes, but how do I figure out how to get an 82\%?
\end{dialogue}

Let's call Devyn's score on midterm 3 as $x$.  According to Riley, Devyn's exam average
would be \[ \frac{74 + 82 + x}{3}. \]

In order to determine what score Devyn needs on his last exam, he and Riley will need to setup and solve an \emph{equation} involving this expression.


\begin{problem}
  Is the following an equation: $\displaystyle \dfrac{74 + 82 + x}{3}$?

  \begin{multipleChoice}
    \choice[correct]{yes}
    \choice{no}
  \end{multipleChoice}

  \begin{feedback}
  Remember that equations need an equals sign.  Without one, it is just an expression.  
  \end{feedback}
\end{problem}


\input{../leveledQuestions.tex}

\end{document}
