\documentclass{ximera}

\input{../preamble.tex}

\outcome{Solve linear equations.}
\outcome{Solve quadratic equations.}
\outcome{Solve equations by factoring.}


\title[Dig-In:]{Equations}
\begin{document}
\begin{abstract}
  We discuss solving equations.
\end{abstract}
\maketitle

An equation is a statement expressing the equality between two quantities.  This means that an equation will always have a single equals sign.


\begin{example}
  From Devyn's question above, $\displaystyle \frac{74 + 84 + x}{3}$ is the average of the three exam 
  scores, with the variable $x$ representing the third unknown exam score.  What does the equation
  $\displaystyle \dfrac{74 + 84 + x}{3} = 82$ mean in this setting?
  
  \begin{explanation}
    This is an equation, stating that the average of Devyn's three exam scores is 82.
  \end{explanation}
\end{example}

Usually, when dealing with an equation, we will be looking for values that make the equation true.
A solution to an equation is a value that, when substituted in for the variables, yield a true number statement.

\begin{example}
	$\displaystyle \dfrac{74 + 84 + 88}{3} = 82$, so $x = 88$ is a solution of the equation
	$\displaystyle \dfrac{74 + 84 + x}{3} = 82$.
\end{example}

When we are asked to \emph{solve an equation}, we are being asked to find all solutions.
Exactly how to do that depends on the particular equation involved.

A \emph{linear equation in $x$} is an equation which is equivalent to one with the form
$a x + b = 0$, where $a$ and $b$ are constants, with $a \neq 0$.


To solve a linear equation, we isolate the $x$-term and divide by the coefficient.
\begin{example}
	Solve the linear equation $\displaystyle \dfrac{x}{3} - 9 = 4\left( 2x + \dfrac{5}{2} \right)$.

	\begin{explanation}
		This doesn't really look like $ax + b = 0$ when written like this.  Let's start by distributing the $4$ into 
		the parentheses, then combining like terms.
		\begin{align*}
			\frac{x}{3} - 9 &= 4\left( 2x + \dfrac{5}{2} \right) \\
			\frac{x}{3} - 9 &= 8x + 10
		\end{align*}
		That looks better.  From here, we'll move all the $x$-terms to the right side of the equation, and all of
		the constant terms to the left.
		\begin{align*}
			- 9 - 10 &= 8x - \frac{x}{3}\\
			-19 &= \frac{23}{3} x \\
			x &= -19 \cdot \frac{3}{23} = -\frac{57}{23}
		\end{align*}
	\end{explanation}	
\end{example}


\begin{problem}
 	Solve the linear equation $\displaystyle \dfrac{4}{5}\left(x+2\right) - 3 = \dfrac{x-1}{2}$. 
 	\begin{hint}
    		It may be easier to clear fractions first.
	\end{hint}
  	\begin{prompt}
    		x = $\answer{3}$.
  	\end{prompt}
\end{problem}



\end{document}
