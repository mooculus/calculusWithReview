\documentclass{ximera}

\input{../preamble.tex}

\outcome{}

\title[Break-Ground:]{How fast was the pen going?}

\begin{document}
\begin{abstract}
  Two young mathematicians think about the rate of change of periodic motion.
\end{abstract}
\maketitle

Check out this dialogue between two calculus students (based on a true
story):

\begin{dialogue}
\item[Devyn] Riley, do you remember your pen bouncing on a spring?
\item[Riley] Sure!  I still have the graph of it's height that we made.  It looked like this:
\begin{image}
      \begin{tikzpicture}
        		\begin{axis}[
            		xmin=0,xmax=13.5,ymin=-1.5,ymax=1.5,
            		axis lines=center,
            		ticks=none,
            		width=6in,
            		height=3in,
            		every axis y label/.style={at=(current axis.above origin),anchor=south},
            		every axis x label/.style={at=(current axis.right of origin),anchor=west},
          		]        
          			\addplot [very thick, penColor, samples=100,smooth, domain=(0:13.5)] {-cos(deg(x))};
        		\end{axis}
 	 \end{tikzpicture}
  \end{image}

\item[Devyn] Did you notice that at the top, the graph looks like it has horizontal tangent lines?
\item[Riley] And at the bottoms, too!
\item[Devyn] Right!  That means the pen was at rest at those instants.  What about at other points, though?
\item[Riley] The steepest slopes look like they happen when the graph crossed the $t$-axis.  That means the pen was
moving fastest at those times.
\item[Devyn] How fast was it going though?  And what about at other times?
\item[Riley] Hmmmm. I'm not sure yet\dots
\end{dialogue}

Let's put some labels on the graph so we can talk about it.
\begin{image}
      \begin{tikzpicture}
        		\begin{axis}[
            		xmin=-1,xmax=13.5,ymin=-1.5,ymax=1.5,
            		axis lines=center,
            		ticks=none,
            		width=6in,
            		height=3in,
            		every axis y label/.style={at=(current axis.above origin),anchor=south},
            		every axis x label/.style={at=(current axis.right of origin),anchor=west},
          		]        
          			\addplot [very thick, penColor, samples=100,smooth, domain=(0:13.5)] {-cos(deg(x))};
  				\draw (0,-1) node {\textbullet};
				\draw ( 1.57, 0 ) node {\textbullet};
				\draw (3.14, 1 ) node {\textbullet};
				\draw (4.71, 0 ) node {\textbullet};
		        		\draw (-0.5,-1) node {A};
				\draw ( 1.4, 0.3 ) node {B};
				\draw (3.14, 1.2 ) node {C};
				\draw (4.8, 0.3 ) node {D};
		\end{axis}
 	 \end{tikzpicture}
  \end{image}
  
\begin{problem}
  At which of these points does the pen have the highest velocity?
  \begin{selectAll}
    \choice{$A$}
    \choice[correct]{$B$}
    \choice{$C$}
    \choice{$D$}
    \choice{None of these.}
  \end{selectAll}
\end{problem}

\begin{problem}
  At which of these points does the pen have the lowest speed? 
  \begin{selectAll}
    \choice[correct]{$A$}
    \choice{$B$}
    \choice[correct]{$C$}
    \choice{$D$}
    \choice{None of these.}
  \end{selectAll}
\end{problem}


\begin{problem}
	Use slopes of tangent lines to plot the graph of the derivative.  Does it look familiar?
\end{problem}
%\input{../leveledQuestions.tex}


\end{document}
