\documentclass{ximera}

\input{../preamble.tex}

\outcome{Know the properties of rational functions.}

\title[Break-Ground:]{Will it divide?}

\begin{document}
\begin{abstract}
  %Two young mathematicians think about the plots of functions.
\end{abstract}
\maketitle

Check out this dialogue between two calculus students (based on a true
story):

\begin{dialogue}
\item[Devyn] Riley, I'm wondering about polynomials.
\item[Riley] What about them?
\item[Devyn] If we have $x^2+x$ and $x^2-4$, I know how to add them.  I know how to subtract them.  I know how to multiply them.  I even know how to factor each of them.
\item[Riley] Sure.  To add or subtract, we combine like terms.  To multiply, we use distribution or FOIL.
\item[Devyn] But we can't just divide $x^2+x$ by $x^2-4$ and arrive at a polynomial!  If we divide polynomials, we don't necessarily end up with another polynomial.
\item[Riley] Sometimes we can, though.  $x^2 + x$ divided by $x$ is basically just the polynomial $x+1$.
\item[Devyn] But if we don't get a polynomial back, what kind of function DO we get?
\end{dialogue}

\begin{problem}
	If we divide $x^3 + 3x^2 + 3x + 1$ by $x+1$, do we end up with a polynomial?
	\begin{multipleChoice}
		\choice[correct]{yes}
		\choice{no}
	\end{multipleChoice}
	\begin{feedback}
		Remember our multiplication formula $(a+b)^3 = a^3 + 3 a^2 b + 3 a b^2 + b^3$.  If we set $a = x$ and $b = 1$, this tells us that
		$x^3 + 3x^2 + 3x + 1$ can be rewritten as $(x + 1)^3$.
	\end{feedback}
\end{problem}

\begin{problem}
	If we divide $x^3 + 3x^2 + 3x + 4$ by $x+1$, do we end up with a polynomial?
	\begin{multipleChoice}
		\choice{yes}
		\choice[correct]{no}
	\end{multipleChoice}
	\begin{feedback}
		We saw above that $x^3 + 3x^2 + 3x + 1$ can be rewritten as $(x + 1)^3$.  That means $x^3 + 3x^2 + 3x + 4 = (x+1)^3 + 3$,
		so $\displaystyle \dfrac{x^3 + 3x^2 + 3x + 4}{x+1} = x^2 + 2x + 1 + \dfrac{3}{x+1}$, which isn't a polynomial.
	\end{feedback}
\end{problem}





\input{../leveledQuestions.tex}


\end{document}
