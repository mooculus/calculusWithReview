\documentclass{ximera}

\input{../../preamble.tex}

\author{Carl Stitz \and Jeff Zeager \and  Bobby Ramsey}
\license{CC-By-SA-NC}
\acknowledgement{http://www.stitz-zeager.com/}

\begin{document}
Consider the rational function $\displaystyle f(x) = \frac{x^3+1}{x^2-1}$.

\begin{exercise}
	How many vertical asymptotes does $f$ have? \\
	
	It has $\answer{1}$.

	\begin{feedback}
		Remember to factor the numerator!
	\end{feedback}	
	\begin{exercise}
		The vertical asymptote is at $x = \answer{1}$.
		\begin{exercise}
			What happens to $f$ at $x=-1$? (Choose all that apply)
			\begin{selectAll}
				\choice[correct]{It is not in the domain of $f$.}
				\choice{It is a zero of $f$.}
				\choice{It is a vertical asymptote of $f$.}
				\choice{It is a horizontal asymptote of $f$.}
				\choice[correct]{It is a hole in the graph of $f$.}
			\end{selectAll}
		\end{exercise}	
	\end{exercise}
\end{exercise}
	
\begin{exercise}
	If you divide $x^3+1$ by $x^2-1$, what is the quotient and remainder?
	\[\text{Quotient} \,\, = \answer{x} \quad \text{Remainder} \,\, = \answer{1}\]
	This means we can rewrite $f$ in terms of proper fractions as:
	\[ f(x) = \answer{x} + \frac{\answer{1}}{x^2-1} \]
	\begin{exercise}
		What is the end behavior of $f$?
		\begin{align*}
			\text{As} \,\, x \to \infty, &\quad f(x) \to \answer{\infty}\\
			\text{As} \,\, x \to -\infty, &\quad f(x) \to \answer{-\infty}\\
		\end{align*}	
	\end{exercise}	
\end{exercise}
	




\end{document}
