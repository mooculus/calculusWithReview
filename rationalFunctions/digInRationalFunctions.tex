\documentclass{ximera}

\input{../preamble.tex}

\outcome{Know the properties of rational functions.}
\outcome{Understand the definition of a rational function.}

\title[Dig-In:]{Working with rational functions}


\begin{document}
\begin{abstract}
  Rational functions are functions defined by fractions of
  polynomials.
\end{abstract}
\maketitle


\section{What are rational functions?}
In algebra, polynomials play the same role as the integers do in arithmetic.  We add them, subtract them, multiply them, and factor them.  We cannot 
divide them, however, if we want an integer answer.  Since $4$ is not a factor of $7$, $\dfrac{7}{4}$ is not an integer.  If we want to be able to divide
integers, we have to move to the \emph{rational numbers}, which are fractions $\dfrac{p}{q}$ where $p$ and $q$ are integers, and $q \ne 0$.

The same idea holds for polynomials.  We can add them, subtract them, multiply them, and factor them.  However, to divide them we have to move
to rational functions.

\begin{definition}
  A \dfn{rational function} in the variable $x$ is a function the form
  \[
  f(x) = \frac{p(x)}{q(x)}
  \]
  where $p$ and $q$ are polynomial functions, and $q$ is not the constant zero function. The domain of a rational
  function is all real numbers except for where the denominator is
  equal to zero.
\end{definition}

\begin{question}
  Which of the following are rational functions?
  \begin{selectAll}
    \choice[correct]{$f(x) = 0$}
    \choice[correct]{$f(x) = \frac{3x+1}{x^2-4x+5}$}
    \choice{$f(x)=e^x$}
    \choice{$f(x)=\frac{\sin(x)}{\cos(x)}$}
    \choice[correct]{$f(x) = -4x^{-3}+5x^{-1}+7-18x^2$}
    \choice{$f(x) = x^{1/2}-x +8$}
    \choice{$f(x)=\frac{\sqrt{x}}{x^3-x}$}
  \end{selectAll}
  \begin{feedback}
    All polynomials can be thought of as rational functions.
  \end{feedback}
\end{question}



\section{Working with rational functions}

\begin{example}
	Find the domain of the rational function $\displaystyle f(x) = \dfrac{4x^3+1}{6x^2-7x-3}$
	\begin{explanation}
		We start by setting the denominator equal to zero.
		\begin{align*}
			6x^2 - 7x - 3 &= 0\\
			(3x + 1)(2x - 3)&= 0\\
			x &= -\dfrac{1}{3} , \dfrac{3}{2}
		\end{align*}
		The domain of $f$ is all $x$ except these two values.  Thus:
		\[ \left( -\infty , \dfrac{1}{3}\right) \bigcup \left( -\dfrac{1}{2}, \dfrac{3}{2} \right) \bigcup \left( \dfrac{3}{2}, \infty \right). \]
	\end{explanation}
\end{example}


When we need to simplify the form of a rational expression, our approach depends on the particular form we are presented with.  If it consists of
only a single fraction, we divide out the common factors.
\begin{example}
	Simplify the expression $\displaystyle \dfrac{2x^2-3x-2}{4x^2+6x+2}$.
	\begin{explanation}
		We begin by factoring.  The numerator factors as $2x^2 - 3x -2 = (2x+1)(x-2)$, and the denominator factors as $4x^2+6x+2 = 2(2x+1)(x+1)$.
		That means,
		\begin{align*}
			\dfrac{2x^2-3x-2}{4x^2 + 6x + 2} &= \dfrac{(2x+1)(x-2)}{2(2x+1)(x+1)} \\
				&= \dfrac{2x+1}{2x+1} \cdot \dfrac{x-2}{2(x+1)}\\
				&= \dfrac{x-2}{2(x+1)},
		\end{align*}	
	\end{explanation}
\end{example}

When dealing with a sum/difference of two fractions, we must first convert to a common denominator.  After the addition,
we can divide away any common factors that are still present.
\begin{example}
	Simplify the expression \[ \dfrac{x-6}{x^2+2x} + \dfrac{x+3}{x^2+x}. \]
	\begin{explanation}
		We'll start by factoring the denominators.  $x^2+2x = x(x+2)$, and $x^2+x = x(x+1)$.
		These two fractions have a common denominator of $x(x+1)(x+2)$.
		\begin{align*}
			\dfrac{x-6}{x^2+2x} + \dfrac{x+3}{x^2+x} &= \dfrac{x-6}{x(x+2)} + \dfrac{x+3}{x(x+1)}\\
				&= \dfrac{(x-6)(x+1)}{x(x+2)(x+1)} + \dfrac{(x+3)(x+2)}{x(x+1)(x+2)}\\
				&= \dfrac{x^2-5x-6}{x(x+2)(x+1)} + \dfrac{x^2+5x+6}{x(x+2)(x+1)}\\ 
				&= \dfrac{2x^2}{x(x+2)(x+1)}\\
				&= \dfrac{2x}{(x+2)(x+1)}.
		\end{align*}
		Once we convert each fraction to the common denominator, we added numerators, then simplified.
	\end{explanation}
\end{example}

If we have a complex fraction, involving a fraction in the numerator or the denominator, we can start by multiplying the numerator
and denominator of the big fraction, by the common denominator of the smaller fractions.  That eliminates the smaller fractions leaving
the outer one to deal with.
\begin{example}
	Simplify the expression \[ \dfrac{ \dfrac{x+2}{x-1} + x}{\dfrac{x}{x-1} + \dfrac{2x+1}{x+1}}.\]
	\begin{explanation}
		The smaller fractions in the numerator and denominator have a common denominator $(x-1)(x+1)$.  We begin my multiplying
		both the numerator and denominator by $(x-1)(x+1)$.
		\begin{align*}
			\dfrac{ \dfrac{x+2}{x-1} + x}{\dfrac{x}{x-1} + \dfrac{2x+1}{x+1}} &= \dfrac{ \dfrac{x+2}{x-1} + x}{\dfrac{x}{x-1} + \dfrac{2x+1}{x+1}} \cdot \dfrac{(x-1)(x+1)}{(x-1)(x+1)}\\
				&= \dfrac{(x+2)(x+1) + x(x+1)(x-1)}{x(x+1) + (2x+1)(x-1)}\\
				&= \dfrac{(x+1)\left( (x+2)  + (x^2-x) \right)}{(x^2+x) + (2x^2 - x - 1)}\\
				&= \dfrac{(x+1)\left( x^2 + 2 \right)}{3 x^2 - 1}
		\end{align*}
		There are no common factors between the numerator and denominator, so this cannot be simplified any further.
	\end{explanation}
\end{example}

\begin{example}
	For the rational function $\displaystyle f(x) = \frac{2x}{x-3}$, find and simplify the following:
		\[ \frac{f(x+h)-f(x)}{h} \]
	\begin{explanation}
		$\displaystyle f(x+h)$ means replace $x$ in the formula for $f$ with $x+h$.  This gives:
		\begin{align*}
			\frac{\frac{2(x+h)}{(x+h)-3} - \frac{2x}{x-3} }{h} &= \frac{ \frac{2x+2h}{x+h-3} - \frac{2x}{x-3}}{h} \\
				&= \frac{ \frac{ (2x+2h)(x-3) }{(x+h-3)(x-3)} - \frac{2x(x+h-3)}{(x-3)(x+h-3)} }{h}\\
				&= \frac{ \frac{2x^2-6x+2xh-6h}{(x+h-3)(x-3)} - \frac{2x^2+2xh-6x}{(x+h-3)(x-3)}  }{h}\\
				&= \frac{  \frac{-6h}{(x+h-3)(x-3)}     }{h} \\
				&= \frac{-6h}{h(x+h-3)(x-3)} \\
				&= \frac{-6}{(x+h-3)(x-3)}.
		\end{align*}
	\end{explanation}
\end{example}
The quantity $\frac{f(x+h)-f(x)}{h}$ is frequently referred to as the \emph{Difference Quotient}.  We'll be seeing much more
of the difference quotient soon.

\end{document}
