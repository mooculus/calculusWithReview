\documentclass{ximera}

\input{../preamble.tex}

\title[Dig-In:]{L'H\^{o}pital's rule}

\outcome{Recall how to find limits for forms that are not indeterminate.}
\outcome{Define an indeterminate form.}
\outcome{Convert indeterminate forms to the form zero over zero or infinity over infinity.}
\outcome{Define L'Hopital's Rule and identify when it can be used.}
\outcome{Use L'Hopital's Rule to find limits.}

\begin{document}
\begin{abstract}
  We use derivatives to give us a ``short-cut'' for computing limits.
\end{abstract}
\maketitle

Derivatives allow us to take problems that were once difficult to
solve and convert them to problems that are easier to solve. Let us
consider L'H\^{o}pital's rule:

\begin{theorem}[L'H\^{o}pital's Rule]\index{L'H\^opital's Rule} 
Let $f(x)$ and $g(x)$ be functions that are differentiable near $a$.  If
\[
\lim_{x \to a} f(x) = \lim_{x \to a}g(x) = 0 \qquad \text{or} \pm \infty,
\]
and $\lim_{x \to a} \frac{f'(x)}{g'(x)}$ exists, and $g'(x) \neq 0$
for all $x$ near $a$, then 
\[
\lim_{x \to a} \frac{f(x)}{g(x)} = \lim_{x \to a} \frac{f'(x)}{g'(x)}.
\]
\end{theorem}

This theorem is somewhat difficult to prove, in part because it
incorporates so many different possibilities, so we will not prove it
here. 
\begin{remark}
  L'H\^{o}pital's rule applies even when $\lim_{x\to a}f(x) = \pm \infty$
  and $\lim_{x\to a}g(x) = \mp \infty$.
\end{remark}


L'H\^{o}pital's rule allows us to investigate limits of
\textit{indeterminate form}.

\begin{definition}[List of Indeterminate Forms]\index{indeterminate form}\hfil
\begin{description}
\item[\zeroOverZero] This refers to a limit of the form $\lim_{x\to a}
  \frac{f(x)}{g(x)}$ where $f(x)\to 0$ and $g(x)\to 0$ as $x\to a$.
\item[\inftyOverInfty] This refers to a limit of the form $\lim_{x\to a}
  \frac{f(x)}{g(x)}$ where $f(x)\to \infty$ and $g(x)\to \infty$ as $x\to a$.
\item[\zeroTimesInfty] This refers to a limit of the form $\lim_{x\to a}
  \left(f(x)\cdot g(x)\right)$ where $f(x)\to 0$ and $g(x)\to \infty$ as $x\to a$.
\item[\inftyMinusInfty] This refers to a limit of the form $\lim_{x\to a}\left(
  f(x)-g(x)\right)$ where $f(x)\to \infty$ and $g(x)\to \infty$ as $x\to a$.
\item[\oneToInfty] This refers to a limit of the form $\lim_{x\to a}
  f(x)^{g(x)}$ where $f(x)\to 1$ and $g(x)\to \infty$ as $x\to a$.
\item[\zeroToZero] This refers to a limit of the form $\lim_{x\to a}
  f(x)^{g(x)}$ where $f(x)\to 0$ and $g(x)\to 0$ as $x\to a$.
\item[\inftyToZero] This refers to a limit of the form $\lim_{x\to a}
  f(x)^{g(x)}$ where $f(x)\to \infty$ and $g(x)\to 0$ as $x\to a$.
\end{description}
In each of these cases, the value of the limit is \textbf{not} immediately
obvious. Hence, a careful analysis is required!
\end{definition}

\section{Basic indeterminant forms}


Our first example is the computation of a limit that was somewhat
difficult before.

\begin{example}
Compute
\[
\lim_{x\to 0} \frac{\sin(x)}{x}.
\]
\begin{explanation}
Set $f(x) = \sin(x)$ and $g(x) = x$.  Since both $f(x)$ and $g(x)$ are
differentiable functions at $0$, and 
\[
\lim_{x \to 0} f(x) = \lim_{x \to 0}g(x) = 0,
\]
this situation is ripe for L'H\^{o}pital's Rule. Now
\[
f'(x) = \answer[given]{\cos(x)}
\]
and
\[
g'(x) = \answer[given]{1}.
\] 
L'H\^{o}pital's rule tells us that 
\[
\lim_{x \to 0} \frac{\sin(x)}{x} = \lim_{x \to 0} \frac{\cos(x)}{1} = 1.
\]
\end{explanation}
\end{example}

\begin{remark}
  Note, the astute mathematician will notice that in our example
  above, we are somewhat cheating. To apply L'H\^ opital's rule, we
  need to know the derivative of sine; however, to know the derivative
  of sine we must be able to compute the limit:
  \[
  \lim_{x\to 0}\frac{\sin(x)}{x}
  \]
  Hence using L'H\^{o}pital's rule to compute this limit is a circular
  argument! We encourage the gentle reader to view L'H\^{o}pital's rule
  a ``reminder'' as to what is true, not as the formal derivation of
  the result.
\end{remark}




\begin{example}
  Compute 
\[
\lim_{x\to \pi/2^+} \frac{\sec(x)}{\tan(x)}.
\]
\begin{explanation}
Set $f(x) = \sec(x)$ and $g(x) = \tan(x)$. Both $f(x)$ and $g(x)$
are differentiable near $\pi/2$. Additionally,
\[
\lim_{x \to \pi/2^+} f(x) = \lim_{x \to \pi/2^+}g(x) = -\infty.
\]
This situation is ripe for L'H\^opital's Rule. Now 
\[
f'(x) = \answer[given]{\sec(x)\tan(x)}
\]
and
\[
g'(x) = \answer[given]{\sec^2(x)}.
\]
L'H\^{o}pital's rule tells us that 
\begin{align*}
\lim_{x\to \pi/2^+} \frac{\sec(x)}{\tan(x)} &= \lim_{x\to \pi/2^+}
\frac{\sec(x)\tan(x)}{\sec^2(x)} \\
&= \lim_{x\to \pi/2^+} \sin(x)\\
&=\answer[given]{1}.
\end{align*}
\end{explanation}
\end{example}


\begin{example}
\[
\lim_{x\to \infty} \frac{e^x+2x+1}{x^2}.
\]
	\begin{explanation}
	As $\lim_{x\to\infty} (e^x+2x+1)=\answer[given]{\infty}$ and 
	$\lim_{x\to\infty} x^2 = \answer[given]{\infty}$,
	this is another $\inftyOverInfty$-form, so L'H\^opital's Rule applies.
	\[ \lim_{x\to\infty}\frac{e^x+2x+1}{x^2}=\lim_{x\to\infty}\frac{\answer[given]{e^x+2}}{\answer[given]{2x}} \]
	
	Since $\lim_{x\to\infty}(e^x+2)=\infty$ and $\lim_{x\to\infty}2x=\infty$, this
	is another $\inftyOverInfty$-form.  We'l use L'H\^opital's Rule again!
	\[\lim_{x\to\infty}\frac{e^x+2}{2x}=\lim_{x\to\infty}\frac{\answer[given]{e^x}}{\answer[given]{2}} = \answer[given]{\infty} \]
	
	\end{explanation}
\end{example}


\end{document}

