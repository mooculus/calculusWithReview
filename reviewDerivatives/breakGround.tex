\documentclass{ximera}

\input{../preamble.tex}

\outcome{Review derivatives.}

\title[Break-Ground:]{Review Derivatives BreakGround}

\begin{document}
\begin{abstract}
  Two young mathematicians think about derivatives.
\end{abstract}
\maketitle

Check out this dialogue between two calculus students (based on a true
story):

\begin{dialogue}
\item[Devyn] Riley, I've got a weird limit I'm trying to evaluate.
\item[Riley] What is it?
\item[Devyn] I'm trying to find \[ \lim_{x \to 0} \frac{(e^x + \sin x) - 1}{x}.\]
\item[Riley] Have you tried just plugging in $0$?
\item[Devyn] Yes, but it looks like a \zeroOverZero form.  Usually I would just do some algebra to simplify the limit and it would work out,
		but there doesn't seem to be any algebra to do.
\item[Riley] Hmmm... I want to try something.  What if we take that $e^x + \sin x$ group and call it $f(x)$.  Then this limit is
		\[ \lim_{x \to 0} \frac{f(x) - 1}{x}.\]
\item[Devyn] Oh!  With that choice of $f$, we know $f(0) = 1$, so I can rewrite that limit as
		\[ \lim_{x \to 0} \frac{f(x) - f(0)}{x-0}.\]
\item[Riley] That looks familiar!
\end{dialogue}

\begin{problem}
 In trying to evaluate Devyn's limit, Riley and Devyn managed to rewrite it into a form they had worked with before.  Why does this limit look familiar?
  \begin{multipleChoice}
  	  \choice{It is a \zeroOverZero-form.}
  	  \choice[correct]{It is the definition of the derivative.}
  	  \choice{It is how you solve optimization problems.}
  	  \choice{It becomes a Squeeze Theorem problem.}
  	  \choice{None of the above.}
  \end{multipleChoice}
\end{problem}

\begin{problem}
Use what you found out about that form to find the value of the Devyn's limit?
	\begin{prompt}
		$\displaystyle \lim_{x \to 0} \frac{(e^x + \sin x) - 1}{x} = \answer{2}$.
	\end{prompt}
\end{problem}



%\input{../leveledQuestions.tex}


\end{document}
