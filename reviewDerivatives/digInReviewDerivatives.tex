\documentclass{ximera}

\input{../preamble.tex}

\outcome{Review derivatives.}

\title[Dig-In:]{Review Derivatives}


\begin{document}
\begin{abstract}
  Definition of the derivative and differentiation rules
\end{abstract}
\maketitle


\section{What are derivatives?}

\begin{definition}
  A \dfn{rational function} in the variable $x$ is a function the form
  \[
  f(x) = \frac{p(x)}{q(x)}
  \]
  where $p$ and $q$ are polynomial functions. The domain of a rational
  function is all real numbers except for where the denominator is
  equal to zero.
\end{definition}

\begin{question}
  Which of the following are rational functions?
  \begin{selectAll}
    \choice[correct]{$f(x) = 0$}
    \choice[correct]{$f(x) = \frac{3x+1}{x^2-4x+5}$}
    \choice{$f(x)=e^x$}
    \choice{$f(x)=\frac{\sin(x)}{\cos(x)}$}
    \choice[correct]{$f(x) = -4x^{-3}+5x^{-1}+7-18x^2$}
    \choice{$f(x) = x^{1/2}-x +8$}
    \choice{$f(x)=\frac{\sqrt{x}}{x^3-x}$}
  \end{selectAll}
  \begin{feedback}
    All polynomials can be thought of as rational functions.
  \end{feedback}
\end{question}


%% \section{Connections to polynomials}
%% $\frac{x^2-3x+2}{x-2} \ne x-1$

\end{document}
