\documentclass{ximera}

\input{../preamble.tex}

\outcome{Review limits.}

\title[Break-Ground:]{Guess the Value}

\begin{document}
\begin{abstract}
  Two young mathematicians think about limits.
\end{abstract}
\maketitle

Check out this dialogue between two calculus students (based on a true
story):

\begin{dialogue}
	\item[Devyn] Riley, I want to play a game!
	\item[Riley]  Ok.  What's the game?
	\item[Devyn] I'm thinking of a function.  You are trying to figure out the output value of my function when the input is $4$.
	\item[Riley] So if I call your function $f$, I win if I can guess $f(4)$, right?  What information do I have to work with?
	\item[Devyn] Right!  You can ask for the output values for three different input values that are not $4$.  Any other input is ok.
	\item[Riley] I'm in!  What's $f(3)$?
	\item[Devyn] When the input is $3$, the output is $6.5$, so $f(3) = 6.5$.  That's one value down, two more.  What do you want to know next?
	\item[Riley] What's $f(3.9)$?
	\item[Devyn] I think I see your strategy.  $f(3.9) = 6.895$.
	\item[Riley] With my last value, I'll ask for $f(3.999)$.
	\item[Devyn] $f(3.999) = 6.9999972$.  Was that enough information for you to guess $f(4)$?
\end{dialogue}

\begin{problem}
What is $f(4)$?
  \begin{multipleChoice}
    \choice{$6.99999999999943$}
    \choice{$7.00000123$}
    \choice{$7$}
    \choice[correct]{Not enough information to tell.}
  \end{multipleChoice}
\end{problem}

\begin{problem}
  Which of the following best describes Riley's strategy to find the value?
  \begin{multipleChoice}
    \choice{Differentiation.}
    \choice{Chain Rule.}
    \choice[correct]{Limit.}
    \choice{Optimization.}
    \choice{None of these.}
  \end{multipleChoice}
\end{problem}


\begin{problem}
  Which of the following properties would have given Riley a better chance at guessing the value?
  \begin{selectAll}
    \choice[correct]{Differentiability.}
    \choice[correct]{Continuity.}
    \choice{The function has an absolute maximum value.}
    \choice{None of these.}
  \end{selectAll}
\end{problem}



%%\input{../leveledQuestions.tex}


\end{document}
