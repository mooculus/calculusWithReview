\documentclass{ximera}

\input{../preamble.tex}

\outcome{Know and use the derivatives of the inverse trigonometric functions.}

\title[Break-Ground:]{Derivatives of inverse trigonometric functions BreakGround}

\begin{document}
\begin{abstract}
  Two young mathematicians think about the plots of functions.
\end{abstract}
\maketitle

Check out this dialogue between two calculus students (based on a true
story):

\begin{dialogue}
	\item[Devyn] Riley, do you remember implicit differentiation?
	\item[Riley] Yes.  We have an equation that defines a function, and ask about the derivative of that implicitly defined function.
	\item[Devyn] Exactly!  If we have something like $x^3 + y^3 = 1$...
	\item[Riley]  This defines $y$ as a function of $x$. 
	\item[Devyn] But we could also use it to define $x$ as a function of $y$!
	\item[Riley] Right!  
	\item[Devyn] And we can find both $\dd[y]{x}$ and $\dd[x]{y}$.
	\item[Riley] Absolutely.  It's almost the same work either way.
	\item[Devyn] How are $\dd[y]{x}$ and $\dd[x]{y}$ related to one another?
	\item[Riley] Hmm....
\end{dialogue}

\begin{problem}
	If we call $y=f(x)$ and $x = g(y)$ as the two functions defined by $x^3+y^3=1$, how are $f$ and $g$ related?  
	\begin{multipleChoice}
		\choice{There is no relationship.}
		\choice{They are opposites.}
		\choice[correct]{They are inverses.}
		\choice{They are reciprocals.}
	\end{multipleChoice}
\end{problem}

\begin{problem}
	What is the basic relationship between $\dd[y]{x}$ and $\dd[x]{x}$?
	\begin{multipleChoice}
		\choice{There is no relationship.}
		\choice{They are opposites.}
		\choice{They are inverses.}
		\choice[correct]{They are reciprocals.}
	\end{multipleChoice}
\end{problem}





%\input{../leveledQuestions.tex}


\end{document}
