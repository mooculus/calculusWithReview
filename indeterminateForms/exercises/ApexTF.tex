\documentclass{ximera}

\input{../preamble.tex}

\author{Gregory Hartman \and Matthew Carr}
\license{Creative Commons 3.0 By-NC}
\acknowledgement{https://github.com/APEXCalculus}

\begin{document}


\begin{exercise}

\tag{true or false}
\tag{continuity}

\textbf{True or False}?
\begin{center}
If $f$ is defined on an open interval containing $c$, and $\lim_{x\to c}f\left({x}\right)$ exists, then $f$ is continuous at $c$.
\end{center}

\begin{prompt}
	\begin{multipleChoice}
		\choice{True}
		\choice[correct]{False}
	\end{multipleChoice}
\end{prompt}
\end{exercise}

\begin{exercise}
\begin{center}
If $f$ is continuous at $c$, then $\displaystyle \lim_{x\to c} f(x)$ exists.
\end{center}

\begin{prompt}
	\begin{multipleChoice}
		\choice[correct]{True}
		\choice{False}
	\end{multipleChoice}
\end{prompt}
\end{exercise}

\begin{exercise}
\begin{center}
If $f$ is continuous on $[a,b]$, then $\displaystyle \lim_{x\to a^-} f(x) = f(a)$.
\end{center}

\begin{prompt}
	\begin{multipleChoice}
		\choice{True}
		\choice[correct]{False}
	\end{multipleChoice}
\end{prompt}
\end{exercise}

\begin{exercise}
\begin{center}
If $f$ is continuous on $[0,1)$ and on $[1,2)$, then $f$ is continuous on $[0,2)$.
\end{center}

\begin{prompt}
	\begin{multipleChoice}
		\choice{True}
		\choice[correct]{False}
	\end{multipleChoice}
\end{prompt}
\end{exercise}



\end{document}
