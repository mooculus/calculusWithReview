\documentclass{ximera}

\input{../preamble.tex}

\outcome{...}

\title[Break-Ground:]{Inv Trig Function BreakGround}

\begin{document}
\begin{abstract}
  Two young mathematicians think about trigonometric functions.
\end{abstract}
\maketitle

Check out this dialogue between two calculus students (based on a true
story):

\begin{dialogue}
\item[Devyn] Riley, do you remember talking about trig equations a while ago?
\item[Riley] Absolutely!  We used trig identities to simplify the equation, then built triangles to find reference angles.
\item[Devyn] Right., but if I have a trig equation like $\displaystyle \sin(x) = \frac{2}{5}$....
\item[Riley] SOH-CAH-TOA, so sine is opposite over hypotenuse.  We build a triangle whose opposite side has length $2$, and whose
	hypotenuse is $5$.  The other side is given by $\sqrt{25-2} = \sqrt{23}$.
\item[Devyn] Yes, but what is the angle!  That's not one of the triangles whose angles I remember.
\item[Riley] Oh no!  What do we do?
\end{dialogue}

\begin{problem}
	Recall that two functions $f$ and $g$ are inverses of one another if both:
	\begin{itemize}
		\item $f\left( g(x) \right) = x$ for all $x$ in the domain of $g$.
		\item $g\left( f(x) \right) = x$ for all $x$ in the domain of $f$.
	\end{itemize}
	What condition do we need in order for a function to have an inverse?
  \begin{selectAll}
    \choice[correct]{It must pass the vertical line test.}
    \choice[correct]{It must pass the horizontal line test.}
    \choice{It must be an even function.}
    \choice{It must be an odd function.}
    \choice[correct]{It must be one-to-one.}
  \end{selectAll}
\end{problem}


\begin{problem}
	Does the function $\sin(x)$ have an inverse?
	\begin{multipleChoice}
    		\choice[correct]{No}
    		\choice{Yes}
	  \end{multipleChoice}
\end{problem}


\begin{problem}
	Is $\sin^{-1}(x)$ the inverse of $\sin(x)$?
	\begin{multipleChoice}
    		\choice[correct]{No}
    		\choice{Yes}
	  \end{multipleChoice}
\end{problem}

\begin{problem}
	Find the solution of $\sin(x) = \frac{2}{5}$ that lies in the interval $\left(0, \frac{\pi}{2}\right)$.
	\[ x = \answer{ \sin^{-1}(\frac{2}{5})} \]
\end{problem}
\begin{hint}
	Leave your answer in terms of $\sin^{-1}$.
\end{hint}
%\input{../leveledQuestions.tex}


\end{document}
