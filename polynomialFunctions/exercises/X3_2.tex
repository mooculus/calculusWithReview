\documentclass{ximera}

\input{../../preamble.tex}

\author{Carl Stitz \and Jeff Zeager \and  Bobby Ramsey}
\license{CC-By-SA-NC}
\acknowledgement{http://www.stitz-zeager.com/}

\begin{document}

\begin{exercise}
	The polynomial $x^2 - 2x - 2$ has a zero at $\displaystyle c = 1-\sqrt{3}$.  What is the other zero?
	\[ \answer{1+\sqrt{3}} \]
\end{exercise}

\begin{exercise}
	The polynomial is $\displaystyle x^3	+ 2x^2-3x-6$ has a real zero at $c = -2$.  Find the other real zeroes. (Enter them in order)
	\[ \answer{-\sqrt{3}}  \quad \text{and} \quad \answer{\sqrt{3}} \]
	\begin{feedback}
		Try dividing the polynomial by $x-c$.  
	\end{feedback}
\end{exercise}

\begin{exercise}
	A polynomial $p$ has zeros at $c=\pm 1$ and $c = \pm 2$.  The leading coefficient of $p(x)$ is $117$.  Find the formula for $p(x)$. (You may leave it in factored form)
	\[ p(x) = \answer{117(x+1)(x-1)(x+2)(x-2)} \]
	\begin{feedback}
		Remember that if $x=c$ is a zero of $p$, then $(x-c)$ is a factor of $p(x)$.
	\end{feedback}
\end{exercise}


\end{document}
