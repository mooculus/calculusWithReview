\documentclass{ximera}

\input{../preamble.tex}

\outcome{Know the graphs and properties of ``famous'' functions.}

\title[Dig-In:]{End behavior}


\begin{document}
\begin{abstract}
  Polynomials are some of our favorite functions. 
\end{abstract}
\maketitle



\section{End behavior of polynomial functions}
We will shortly turn our attention to graphs of polynomial functions, but we have one more topic to discuss \emph{End Behavior}.
Basically, we want to know what happens to our function as our input variable $x$ gets really, really large in either the positive or negative direction.
This is kind of like the limits we talked about before, except $x$ is not approaching a fixed value $a$, but just going off to either the right or the left.

We'll start with the most basic polynomials, the \emph{monomials}, $x^n$.  (Since \emph{mono-} means one, the monomials are the polynomials with a single term.)
Here are the graphs of $f(x)=x^n$ for $n = 2$, $4$, and $6$:
	\begin{center}
		\begin{tikzpicture}
        			\begin{axis}[
          			domain=-2:2,
          			xmin=-2, xmax=2,
          			ymin=-8, ymax=8,
          			width=4in,
          			axis lines =middle, xlabel=$x$, ylabel=$y$,
          			every axis y label/.style={at=(current axis.above origin),anchor=south},
          			every axis x label/.style={at=(current axis.right of origin),anchor=west},
          			]
	 	 		\addplot [very thick, penColor, smooth] {x^2};
				\addplot [very thick, penColor2, smooth] {x^4};
				\addplot [very thick, penColor4, smooth] {x^6};
        			\end{axis}
      		\end{tikzpicture}
	\end{center}

	
Here are the graphs of $f(x)=x^n$ for $n = 3$, $5$, and $7$:
	\begin{center}
		\begin{tikzpicture}
        			\begin{axis}[
          			domain=-2:2,
          			xmin=-2, xmax=2,
          			ymin=-8, ymax=8,
          			width=4in,
          			axis lines =middle, xlabel=$x$, ylabel=$y$,
          			every axis y label/.style={at=(current axis.above origin),anchor=south},
          			every axis x label/.style={at=(current axis.right of origin),anchor=west},
          			]
	 	 		\addplot [very thick, penColor, smooth] {x^3};
				\addplot [very thick, penColor2, smooth] {x^5};
				\addplot [very thick, penColor4, smooth] {x^7};
        			\end{axis}
      		\end{tikzpicture}
	\end{center}

Notice the similarity between the all the graphs with $n$ even.  They all have the same basic cup shape, but higher values of $n$ make the graphs flatter in $(-1,1)$ and steeper outside $(-1,1)$.

The same thing happens with the $n$ odd graphs.  They have the same basic shape, but higher values of $n$ make the graphs flatter near the origin and steeper past $1$ and $-1$.

When $n$ is even, what is happening to the output values of $f(x) = x^n$ as $x$ gets larger and larger?  They themselves get larger and larger!  As $x$ increases without bound, so do the outputs.
The same thing happens as $x$ gets larger and larger in the negative direction without bound.  This is the end behavior we were looking for.  We'll say it this way:

\begin{align*}
	\text{As} \quad x \to \infty \, ,  \quad & x^n \to \infty \\
	\text{As} \quad x \to -\infty \, , \quad & x^n \to \infty \quad \text{, for $n$ even}
\end{align*}

For $n$ odd we have a slightly different end behavior.
\begin{align*}
	\text{As} \quad x \to \infty \, ,  \quad & x^n \to \infty \\
	\text{As} \quad x \to -\infty \, , \quad & x^n \to -\infty \quad \text{, for $n$ odd}
\end{align*}

Next we need to see how a coefficient could change this.  Remember how multiplying by a constant transforms a graph?  If the constant is positive the graph is vertically stretched/compressed.
If the constant is negative the graph is flipped over the $x$-axis and then vertically stretched/compressed.  This means if the coefficient of $x^n$ is positive, the end behavior is unaffected.  If the 
coefficient is negative, the end behavior is negated as well.

\begin{example}
	Find the end behavior of $f(x) = -3x^4$.
	\begin{explanation}
		Since $4$ is even, the function $x^4$ has end behavior 
		\begin{align*}
			\text{As} \quad x \to \infty \, ,  \quad & x^4 \to \infty \\
			\text{As} \quad x \to -\infty \, , \quad & x^4 \to \infty 
		\end{align*}
		The coefficient is negative, changing our end behavior to
		\begin{align*}
			\text{As} \quad x \to \infty \, ,  \quad & -3x^4 \to -\infty \\
			\text{As} \quad x \to -\infty \, , \quad & -3x^4 \to -\infty 
		\end{align*}
	\end{explanation}
\end{example}

\begin{problem}
	Find the end behavior of $g(x) = -6 x^9$.
	\begin{align*}
		\text{As} \quad x \to \infty \, ,  \quad & -6x^9 \to \answer{-\infty} \\
		\text{As} \quad x \to -\infty \, , \quad & -6x^9 \to \answer{\infty} 
	\end{align*}
\end{problem}


We understand monomials.  What about more general polynomials? 
\begin{example}
	Find the end behavior of $f(x) = 3x^5 - 4x^3 + 1$.
	\begin{explanation}
		If we rewrite the function as:
		\begin{align*}
			f(x) &= 3x^5 - 4x^3 + 1 \\
				&= x^5 \left( \frac{3x^5}{x^5} - \frac{4x^3}{x^5} + \frac{1}{x^5} \right)\\
				&=x^5 \left( \frac{3x^5}{x^5} - \frac{4x^3}{x^5} + \frac{1}{x^5} \right)\\
				&= x^5 \left( 3 - \frac{4}{x^2} + \frac{1}{x^5} \right)
		\end{align*}
		Remember that as $x\to \pm\infty$ we saw that $x^2 \to \infty$ and that $x^5 \to \mp \infty$.  That means $\frac{4}{x^2} \to 0$ and $\frac{1}{x^5} \to 0$. 
		(We'll talk about this more when we talk about Rational Functions.)  This means everything in parentheses will basically be just $3$ as $x \to \pm \infty$.
		That is, the polynomial $3x^5 - 4x^3 + 1$ has the same end behavior as $3x^5$.  Thus
		\begin{align*}
			\text{As} \quad x \to \infty \, ,  \quad & f(x) \to \infty \\
			\text{As} \quad x \to -\infty \, , \quad & f(x) \to -\infty 
		\end{align*}
	\end{explanation}
\end{example}

This example showed us that \emph{the end behavior of a polynomial is the same as the end behavior of its leading term}.
\begin{problem}
	Find the end behavior of $g(x) = -6x^9 + 15x^5 + 7 x^4 - 18 x^3 + 91x^2 - 72 x + 4$
	\begin{align*}
		\text{As} \quad x \to \infty \, ,  \quad & g(x) \to \answer{-\infty} \\
		\text{As} \quad x \to -\infty \, , \quad & g(x) \to \answer{\infty} 
	\end{align*}
\end{problem}




\end{document}
