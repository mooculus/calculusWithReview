\documentclass{ximera}

\input{../preamble.tex}

\outcome{Know the derivatives of exponential and logarithmic functions.}

\title[Break-Ground:]{Interesting changes}

\begin{document}
\begin{abstract}
  Two young mathematicians think about the plots of functions.
\end{abstract}
\maketitle

Check out this dialogue between two calculus students (based on a true
story):

\begin{dialogue}
\item[Devyn] Riley, remember when I asked you about compound interest?
\item[Riley] Yes I do!  You were dealing with a formula like \[ A = P\left( 1 + \dfrac{r}{n} \right)^{nt}.\]
\item[Devyn] I was.  With monthly compoundings and 3\% interest with a principal of \$1, it became just
			\[ A = (1.0025)^{12t}.\]
\item[Riley] I remember that. What about it?
\item[Devyn] As time passes, the account is worth more and more.
\item[Riley] Makes sense.  You are gaining interest each month.
\item[Devyn] But by how much?  How fast is it growing?
\item[Riley] You are thinking about rates of change!  
\item[Devyn] Right!  How do we find the derivative $\ddt{A}$?
\item[Riley] It isn't a polynomial, so we can't use Power Rule.  What can we do?

\end{dialogue}

\begin{problem}
  What is the AVERAGE rate of change of \[A = (1.0025)^{12t}\] over the first 2 years (so from $t=0$ to $t=2$)?  (ROUND TO 2 DECIMAL PLACES)
  \begin{multipleChoice}
    \choice[correct]{$0.03$.}
    \choice{$0.06$.}
    \choice{$1.06$.}
    \choice{$1$.}
  \end{multipleChoice}
\end{problem}

\begin{problem}
  What are the units of that average rate of change?
    \begin{multipleChoice}
    \choice{dollars}
    \choice{years}
    \choice[correct]{dollars/year}
    \choice{years/dollar}
    \choice{It does not have a unit}
  \end{multipleChoice}
\end{problem}


%\input{../leveledQuestions.tex}


\end{document}
