\documentclass{ximera}

\input{../preamble.tex}

\outcome{Demonstrate that we can only sometimes find a limit value by 
	finding the limit of quotient of the derivative of the numerator and 
	derivative of the denominator.}
\outcome{Explore the hypothesis of L'Hospital's Rule.}
\outcome{Make a guess and test this guess.}


\title[Break-Ground:]{Indeterminate mutterings}

\begin{document}
\begin{abstract}
%Here we see a dialogue where limits are computed using derivatives.
Two young mathematicians consider a way to compute limits using derivatives.
\end{abstract}
\maketitle

Check out this dialogue between two calculus students (based on a true
story):

\begin{dialogue}
\item[Devyn] Riley! I think I have a problem!
\item[Riley] What kind?
\item[Devyn] I was calculating some limits last night.
\item[Riley] Fun times.
\item[Devyn] I realized something.  L'H\^opital's Rule lets us deal with
	$\zeroOverZero$ and $\inftyOverInfty$ forms. 
\item[Riley] Right!
\item[Devyn] But what about all the other indeterminate forms?  Like this one:
 	$\displaystyle \lim_{x\to 0^+} \left( \csc(x)(e^x-1)\right) $.
\item[Riley] So $\lim_{x\to0^+} \csc(x) = \infty$ and $\lim_{x\to0^+}e^x-1 = 0$.
	This is a $\zeroTimesInfty$-form, not $\zeroOverZero$ or $\inftyOverInfty$.
\item[Devyn] EXACTLY!  L'H\^opital's Rule only works on those fraction forms, not
	on this product form.
\item[Riley] Can you take $\csc(x)(e^x-1)$ and write it like a fraction?
\item[Devyn] Oh, like how $\csc(x) = \frac{1}{\sin(x)}$?
\item[Riley] Yes, because then $\csc(x)(e^x-1) = \frac{e^x-1}{\sin(x)}$.
\item[Devyn] Great!  That means my limit is the same as 
	$\lim_{x\to0^+} \frac{e^x-1}{\sin(x)}$ which is a $\zeroOverZero$-form!
\end{dialogue}

\begin{problem}
	By L'H\^opital's Rule,
	\[ \lim_{x\to0^+}\frac{e^x-1}{\sin(x)} = 
	\lim_{x\to0^+} \frac{\answer{e^x}}{\answer{\cos(x)}} = 
	\answer{1}\]
\end{problem}

\begin{problem}
	Write the function $e^x$ as a fraction with a numerator of $1$.
	\[ e^x = \frac{1}{\left(\answer{e^{-x}}\right)} \]	
\end{problem}

\begin{problem}
	Write the function $x+3$ as a fraction with a numerator of $1$.
	\[ x+3 = \frac{1}{\left(\answer{\frac{1}{x+3}}\right)} \]	
\end{problem}
%\input{../leveledQuestions.tex}


\end{document}
