\documentclass{ximera}

\input{../../preamble.tex}

\author{Bobby Ramsey}

\outcome{Define net area.}
\outcome{Approximate net area.}

\begin{document}


Let $f$ be the piecewise linear function given in the following graph.
		\begin{center}
			\begin{tikzpicture}
					\begin{axis}[
						xmin=-1.5, xmax=4.5, ymin=-2,ymax=2,    
						axis lines =middle, 
						every axis y label/.style={at=(current axis.above origin),anchor=south},
						every axis x label/.style={at=(current axis.right of origin),anchor=west},
						xtick={-1,...,6}, ytick={-2,...,8},
						grid=major, width=5in, height=4in,
						]
						\addplot[color=blue, very thick, smooth, domain=-1:0]{-1};	
						\addplot[color=blue, very thick, smooth, domain=0:2]{x-1};							
						\addplot[color=blue, very thick, smooth, domain=2:3]{1};	
						\addplot[color=blue, very thick, smooth, domain=3:4]{4-x};	
					\end{axis}
			\end{tikzpicture}
		\end{center}


\begin{exercise}
	Find the net area on each of the following intervals.

	\begin{align*}
		[-1,0] & \implies \answer{-1} \\ \\
		[-1,1] & \implies \answer{1/2} \\ \\
		[0,2]  & \implies \answer{0} \\ \\
		[-1,2] & \implies \answer{0}\\ \\
		[1,4] & \implies \answer{2}
	\end{align*}
	\begin{hint}
		Net area means area above the $x$-axis minus the area below the $x$-axis.
	\end{hint}	
\end{exercise}

\end{document}
