\documentclass{ximera}

\input{../preamble.tex}

\outcome{}

\title[Break-Ground:]{Computing areas}

\begin{document}
\begin{abstract}
Two young mathematicians discuss cutting up areas.
\end{abstract}
\maketitle

Check out this dialogue between two calculus students (based on a true
story):

%% Definite integral properties, (odd fxns), breaking up intereval


\begin{dialogue}
	\item[Devyn] So, Riemann sums compute approximate areas, but we have to pick sample points first.
	\item[Riley] Right.
	\item[Devyn] If we change sample points, what happens?
	\item[Riley] We get different areas.
	\item[Devyn] Is the area of the region changing?  The region isn't changing.  Why does the Riemann sum change value?
\end{dialogue}

\begin{problem}
	If we change sample points, does the area of the region change?
  \begin{multipleChoice}
	\choice{Yes}
	\choice[correct]{No}
  \end{multipleChoice}
\end{problem}

\begin{problem}
	When we change sample points, does the value of the Riemann sum change?
  \begin{multipleChoice}
	\choice[correct]{Yes}
	\choice{No}
  \end{multipleChoice}
\end{problem}

\begin{problem}
	If the region's area does not change, why is the Riemann sum changing?
	\begin{freeResponse}
		Answer will vary.
	\end{freeResponse}
\end{problem}



%\input{../leveledQuestions.tex}


\end{document}
