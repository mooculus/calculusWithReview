\documentclass{ximera}

\input{../preamble.tex}

\outcome{}

\title[Break-Ground:]{Follow the bouncing pen.}

\begin{document}
\begin{abstract}
  Two young mathematicians think about periodic motion.
\end{abstract}
\maketitle

Check out this dialogue between two calculus students (based on a true
story):

\begin{dialogue}
\item[Devyn] Hey Riley, what is that?
\item[Riley] I hooked a weak spring to the bottom of this shelf, then hung a pen on it.  I'm watching the pen bounce up and down.
\item[Devyn] Sure, I can see that.
\item[Riley] See how the pen starts down at the bottom, bounces up to the top, then back to the bottom again?
\item[Devyn] Yeah.  It's just repeating that same pattern over and over again.
\item[Riley] I graphed the height of that pen, with respect to time.   Here's what I found.
 \begin{image}
      \begin{tikzpicture}
        		\begin{axis}[
            		xmin=0,xmax=13.5,ymin=-1.5,ymax=1.5,
            		axis lines=center,
            		ticks=none,
            		width=6in,
            		height=3in,
            		every axis y label/.style={at=(current axis.above origin),anchor=south},
            		every axis x label/.style={at=(current axis.right of origin),anchor=west},
          		]        
          			\addplot [very thick, penColor, samples=100,smooth, domain=(0:13.5)] {-cos(deg(x))};
        		\end{axis}
 	 \end{tikzpicture}
  \end{image}
\item[Devyn] Hey, that looks familiar!
\end{dialogue}

\begin{problem}
  Devyn and Riley have discovered simple harmonic motion.  What function did Riley plot?
   \begin{multipleChoice}
    \choice{$\sin t$}
    \choice{$\cos t$}
    \choice{$\tan t$}
    \choice{$-\sin t$}
    \choice[correct]{$-\cot t$}
    \choice{$-\tan t$}
  \end{multipleChoice}
\end{problem}



\begin{problem}
   Since we're talking about trigonometric functions, we have to be very clear about our units.  What units do we use to measure angles / rotation when we see an expression like $\sin \theta$?
    \begin{multipleChoice}
    	\choice{Only degrees.}
	\choice{Usually degrees, but occasionally radians.}
	\choice{Occasionally degrees, but usually radians.}
        \choice[correct]{Only radians.}
     \end{multipleChoice}
\end{problem}



%\input{../leveledQuestions.tex}


\end{document}
