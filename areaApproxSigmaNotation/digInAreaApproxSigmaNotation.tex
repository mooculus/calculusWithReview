\documentclass{ximera}

\input{../preamble.tex}

\outcome{Write a Riemann sum in sigma notation.}
\outcome{Use sigma notation to calculate area approximations.}

\title[Dig-In:]{Area approximations with sigma notation}

\begin{document}
\begin{abstract}
\end{abstract}
\maketitle

Remember the Riemann sum, written as:
\[ f(x_1^*) \Delta x + f(x_2^*) \Delta x + \ldots + f(x_n^*) \Delta x\]

The only change from one term to the next, is the subscript of the sample point.
That subscript runs from $1$ to $n$.  That means we can write this Riemann sum
in sigma notation as:
\[ \sum_{k=1}^n f(x_k^*) \Delta x \]

We found formulas for the sample points in certain cases.

The left Riemann sum is
\[ \sum_{k=1}^n f(a + (k-1)\Delta x) \Delta x \]

The right Riemann sum is
\[ \sum_{k=1}^n f(a + k \Delta x) \Delta x \]

The midpoint Riemann sum is
\[ \sum_{k=1}^n f\left(a + \left(k-\frac{1}{2}\right)\Delta x\right) \Delta x \]
  
  
\begin{example}
	Approximate the area under the graph of $f(x) = x^2+1$ on the interval $[2,4]$ using $100$ rectangles and right-hand endpoints.
	\begin{explanation}
		Let's start by finding $\Delta x$.  
		
		$\Delta x = \frac{b-a}{n} = \frac{1}{50}$.


		For right-hand endpoints, $x_k^* = a + k \Delta x$.  
		That is, $x_k^* = \answer{2} + k \answer{1/50}$.
		
		The Riemann sum is:
		\begin{align*}
			\sum_{k=1}^{100} f\left( x_k^* \right) \Delta x
				&=\sum_{k=1}^{100} f\left( 2 + \frac{k}{50} \right) \frac{1}{50}\\
				&=\sum_{k=1}^{100}\left( \left( 2 + \frac{k}{50} \right)^2+1 \right)\frac{1}{50}\\
				&= \sum_{k=1}^{100} \left( \frac{k^2}{50^2} + \frac{2}{25}k+5\right) \frac{1}{50}\\
				&= \frac{1}{50}\sum_{k=1}^{100} \left( \frac{k^2}{50^2} + \frac{2}{25}k+5\right)\\
				&= \frac{1}{50}\left(\frac{1}{50^2}\sum_{k=1}^{100} k^2 + \frac{2}{25}\sum_{k=1}^{100}k+ \sum_{k=1}^{100}5\right)
		\end{align*}

		We have formulas for calculating the individual sums here!
		
		\begin{align*} 
			\sum_{k=1}^{100} k^2 &= \frac{(100)(100+1)(2\cdot100+1)}{6} \\ \\
				&= \answer{338350}\\
			\sum_{k=1}^{100} k &= \frac{(100)(100+1)}{2} \\ \\
				&= \answer{5050}\\
			\sum_{k=1}^{100} 5 &= 5 \cdot 100 \\
				&= \answer{500}
		\end{align*}
		
		Then
		\begin{align*}
			\sum_{k=1}^{100} f\left( x_k^* \right) \Delta x
				&=\frac{1}{50}\left(\frac{1}{50^2}\sum_{k=1}^{100} k^2 + \frac{2}{25}\sum_{k=1}^{100}k+ \sum_{k=1}^{100}5\right) \\
				&= \frac{1}{50}\left(\frac{1}{50^2}\answer{338350} + \frac{2}{25}\answer{5050}+ \answer{500}\right) \\
				&= \answer{62/3}
		\end{align*}
	\end{explanation}
\end{example}
 
The process used in this example is the same for almost all of these approximations.  Plug $\Delta x$ and $x_k^*$ into the Riemann sum formula and simplify,
then use our summation formulas.
  
 
 
\end{document}
