\documentclass{ximera}

\input{../preamble.tex}

\outcome{}


\title[Break-Ground:]{So many rectangles.}

\begin{document}
\begin{abstract}
A dialogue where students discuss area approximations.
\end{abstract}
\maketitle

Check out this dialogue between two calculus students (based on a true story):

\begin{dialogue}
\item[Devyn] Hey Riley, I've found a problem with approximating areas!
\item[Riley] Fascinating!  Tell me more.
\item[Devyn] Remember how it works.  We split the region into some number $n$ of 
				rectangles, find the area of each of the rectangles, then add
				them all together.
\item[Riley] Yes, and by taking more rectangles we can make that approximation
				better and better!
\item[Devyn] Pretty much.  But think about the calculations we've been doing.  
				For each rectangle, we're tracking the height and area, then
				adding them all together.  That works for, like, 6 or 8 
				rectangles, but we want LOTS of rectangles to get a really
				good approximation.
\item[Riley] How can we track this for 100 rectangles?!?!?
\item[Devyn] Exactly!
\item[Riley] Hmmm...
\end{dialogue}

\begin{problem}
	Which of the following quantities does the accuracy of our approximation depend upon?
	\begin{selectAll}
		\choice[correct] The function.
		\choice[correct] The interval.
		\choice[correct] The number of rectangles.
	\end{selectAll}
\end{problem}


%%\input{../leveledQuestions.tex}

\end{document}
