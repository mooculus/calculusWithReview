\documentclass{ximera}

\input{../../preamble.tex}

\outcome{Understand the relationship between area under a curve and sums of rectangles.}
\outcome{Approximate area under a curve.}
\outcome{Compute Riemann sums with sigma notation.}

\author{Nela Lakos \and Kyle Parsons \and Bobby Ramsey}

\begin{document}


Consider the function
\[
f(x) = 25 - x^2
\]
on the interval $[0,5]$.  We will approximate the area of the region bounded by the graph of $f$ and the $x$-axis, using
right endpoints and $n=100$ rectangles.

\begin{exercise}
	Find $\Delta x$.
	\[ \Delta x = \answer{1/20} \]
	\begin{hint}
		$\Delta x = \frac{b-a}{n}$
	\end{hint}
	\begin{exercise}
		The sample-points $x_k^*$, for $k = 1, 2, \ldots, n$, are given by:
		\[ x_k^* = \answer{k/20} \]
		\begin{hint}
			Right-hand endpoints have the form $x_k^* = a + k \Delta x$.
		\end{hint}	
		\begin{exercise}
			What is $f( x_k^* )$?
			\[ f\left( \frac{k}{20} \right) = \answer{25-\frac{k^2}{400}} \]
			\begin{hint}
				You have already found that $x_k^* = \frac{k}{20}$.  What do you get if you plug that into $f$?
			\end{hint}
			\begin{exercise}
				The Riemann sum to approximate area is given by
				\[ \sum_{k=1}^{n} f(x_k^*) \Delta x = \sum_{k=1}^{100} \left( \left( 25 - \frac{k^2}{400}\right) \frac{1}{20} \right)\]
				Evaluate this sum. (Exact value)
				\[ \sum_{k=1}^{n} f(x_k^*) \Delta x = \answer{13233/160} \]
				\begin{hint}
					Remember the formulas:
					\begin{align*}
						\sum_{k=1}^{n}  C &= nC\\ \\
						\sum_{k=1}^{n}  k &= \frac{n(n+1)}{2}\\ \\
						\sum_{k=1}^{n}  k^2 &= \frac{n(n+1)(2n+1)}{6}\\ \\
						\sum_{k=1}^{n}  k^3 &= \left( \frac{n(n+1)}{2}\right)^2
					\end{align*}
				\end{hint}
			\end{exercise}			
		\end{exercise}
	\end{exercise}

\end{exercise}
\end{document}