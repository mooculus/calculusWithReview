\documentclass{ximera}

\input{../../preamble.tex}

\outcome{Approximate area under a curve.}
\outcome{Compute Riemann sums with sigma notation.}

\author{Bobby Ramsey}

\begin{document}


Consider the function
\[
f(x) = x-x^2
\]
on the interval $[0,1]$.  Approximate the area of the region bounded by the graph of $f$ and the $x$-axis, using
right endpoints and $n=30$ rectangles.  (Give the exact value of the Riemann sum.)


\begin{exercise}
\[ \text{Right Riemann Sum }\, = \answer{899/5400} \]

\begin{hint}
	\begin{exercise}
	Find $\Delta x$.
	\[ \Delta x = \answer{1/30} \]

	\begin{exercise}
		The sample-points $x_k^*$, for $k = 1, 2, \ldots, n$, are given by:
		\[ x_k^* = \answer{k/30} \]

		\begin{exercise}
			What is $f( x_k^* )$?
			\[ f\left( \frac{k}{30} \right) = \answer{\frac{k}{30}-\frac{k^2}{900}} \]

			\begin{exercise}
				The Riemann sum to approximate area is given by
				\[ \sum_{k=1}^{n} f(x_k^*) \Delta x = \sum_{k=1}^{30} \left( \left( \frac{k}{30} - \frac{k^2}{900}\right) \frac{1}{30} \right)\]
				Evaluate this sum. (Exact value)

			\end{exercise}
		\end{exercise}
	\end{exercise}
	\end{exercise}
\end{hint}
\end{exercise}
\end{document}